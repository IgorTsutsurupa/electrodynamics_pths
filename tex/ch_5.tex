\section{Экспериментальные газовые законы и их обобщения}

	Как мы уже записали в ~(\ref{2}), 
		$$P\frac{V}{N}=\frac{2}{3}\average{E}_\text{кин}.$$
	Назовем кинетической температурой меру средней кинетической энергии молекул газа, т.е. $\average{E}_\text{кин}=\c \cdot T_{\text{кин}}$ и, далее,
		$$\average{E}_\text{кин}=\frac{3}{2}kT.$$
	Тогда ~(\ref{2}) можно переписать в виде
	\begin{equation}\label{3}
		p=\frac{N}{V}kT=nkT.
	\end{equation}
	Концентрацию газа $n$ можно записать так:
		$$n=\frac{N}{V}=\frac{1}{V}\frac{m}{\mu}N_A,$$
	где $m$ -- масса газа, $\mu$ -- его молярная масса, $V$ -- его объем. Тогда, подставляя правую часть этого равенства в ~(\ref{3}) и заменяя $kN_A$ на $R$, получим уравнение состояния для произвольной массы идеального газа.
	\begin{equation}\label{4}
		\boxed{PV=\frac{m}{\mu}RT.}
	\end{equation}
	Это уравнение также называют уравнением Менделеева-Клапейрона. Единственная величина в этом уравнении, являющееся характеристикой непосредственно вещества газа, --- это его молярная масса $\mu$. Заметим, что величина $\dfrac{PV}{T}$ остается постоянной при любом состоянии данного газа, а $\dfrac{PV}{\nu T}$ --- для любого газа вообще.\par
	Перейдем к изучению т.н. газовых законов. \textit{Изопроцессом} назовем процесс, протекающий при неизменном значении одного из параметров. Так, например, одним из трех изопроцессов является \textit{изотермический процесс} --- процесс изменения состояния термодинамической системы макроскопических тел при постоянной температуре. Согласно уравнению ~(\ref{4}) в любом состоянии с неизменной температурой произведение газа на его объем остается постоянным:
	\begin{equation}\label{boyle}
		PV=\c \,\, \text{при} \,\, T=\c.
	\end{equation}
	\textit{\textbf{Для газа данной массы произведение давления на его объем постоянно, если его температура не меняется}}. Этот закон называют \textit{законом Бойля-Мариотта}.\par
	Процесс изменения состояния термодинамической системы при постоянном давлении называют \textit{изобарным}. Согласно ~(\ref{4}) в любом состоянии газа с неизменным давлением отношение объема газа к его температуре остается постоянным:
	\begin{equation}\label{lussac}
		\frac{V}{T}=\c \,\, \text{при} \,\, p=\c.
	\end{equation}
	\textit{\textbf{Для газа данной массы отношение его объема к температуре постоянно, если его давление не меняется}}. Этот закон называют \textit{законом Гей-Люссака}. Согласно ~(\ref{lussac}), объем газа при изобарном процессе линейно зависит от температуры:
		$$V=\c\cdot T,$$
	и, вообще говоря,
		$$V=V_0+\alpha V_0 T=V_0(1+\alpha T),$$
	где $\alpha=\dfrac{1}{273,15} \, \dfrac{1}{\kelv}$ -- коэффициент относительного температурного расширения. \par
	Наконец, процесс изменения состояния термодинамической системы при постоянном объем называют \textit{изохорным}. Тогда из ~(\ref{4}) в любом состоянии газа с неизменным объемом отношение давления газа к его температуре остается постоянным:
	\begin{equation}\label{charles}
		\frac{P}{T}=\c \,\, \text{при} \,\, V=\c.
	\end{equation}
	\textit{\textbf{Для газа данной массы отношение его давления к температуре постоянно, если его объем не меняется}}. Этот закон называют \textit{законом Шарля}. Аналогично, тогда, согласно ~(\ref{charles}), давление газа при изохорном процессе линейно зависит от температуры:
		$$P=\c \cdot T,$$
	или
		$$P=P_0(1+\alpha T).$$
%\begin{itemize}
%		\item Газы способны неограниченно расширяться и занимать любой предоставленный объём%
%		\item Смесь газов оказывает давление на стенки сосуда, равное сумме давлений каждого из газа, взятых в отдельности.
%		\item При постоянной температуре давление данной массы газа обратно пропорционально его объёму (Закон Бойля — Мариотта) $PV = const$
%		\item При постоянном объёме  давление данной массы газа линейно зависит от температуры (закон Шарля) т.е. $P= P_0(1+ \alpha T)$, где $T$ - это температура в градусах Цельсия, $P$ – это давление,$P_0$ – давление при $0\degree С$, $\alpha$ – температурный коэффициент равный $1/273.15$
%		\item При постоянном давлении объём данной массы газа линейно зависит от температуры ( закон Гей-Люссака) т.е. V = V0(1+ aT), где V – объём газа при температуре T, V0 – объём газа 00С, a – температурный коэффициент равный 1/273.15
%	\end{itemize}
%\par
%$PV= NkT$
%\par
%$PV=\nu N_akT$
%\par
%$PV=\nu  RT$
