\section{Давление газа согласно молекулярно- \, кинетической теории. Идеальный газ}
У разреженного газа расстояние между молекулами во много раз превышает их размеры. В этом случае взаимодействие между молекулами пренебрежимо мало и кинетическая энергия молекул много больше потенциальной энергии взаимодействия. Вместо реального газа, между молекулами которого действуют сложные силы взаимодействия, мы будем рассматривать его физическую модель, которую назовем идеальным газом. 
\begin{defn}
	Идеальный газ --- газ такой, что размерами его молекул и потенциальной энергией их взаимодействия можно пренебречь.
\end{defn}
Особенностью идеального газа является то, что его внутренняя энергия пропорциональна абсолютной температуре и не зависит от объема, занимаемого газом. Как и во всех случаях использования физических моделей, применимость модели идеального газа к реальному газу зависит не только от свойств самого газа, но и от характера вопроса, на который требуется найти ответ. Такая модель не позволяет описать особенности поведения различных газов, но выделяет общие свойства для всех газов. Идеальный газ удовлетворяет условиям:
\begin{itemize}
	\item объемом всех молекул газа можно пренебречь по сравнению с объемом сосуда, в котором находится газ
	\item время столкновения молекул друг с другом пренебрежимо мало по сравнению со временем между двумя столкновениями (временем свободного пробега)
	\item молекулы взаимодействуют между собой только при непосредственном соприкосновении, при этом они отталкиваются
	\item силы притяжения между молекулами идеального газа пренебрежимо малы
\end{itemize}
Используя эту модель, найдем давление газа на стенку сосуда. Пусть в сосуде произвольной формы находится идеальный газ, состоящий из молекул одного вещества, причем его концентрация известна и равна $n=\dfrac{N}{V}$, где $N$ --- количество молекул газа в сосуде, $V$ --- его объем.