\section{Давление газа согласно молекулярно- \, кинетической теории. Идеальный газ}

	У разреженного газа расстояние между молекулами во много раз превышает их размеры. В этом случае взаимодействие между молекулами пренебрежимо мало и кинетическая энергия молекул много больше потенциальной энергии взаимодействия. Вместо реального газа, между молекулами которого действуют сложные силы взаимодействия, мы будем рассматривать его физическую модель, которую назовем идеальным газом. 
	\begin{defn}
		Идеальный газ --- газ такой, что размерами его молекул и потенциальной энергией их взаимодействия можно пренебречь.
	\end{defn}
	Особенностью идеального газа является то, что его внутренняя энергия пропорциональна абсолютной температуре и не зависит от объема, занимаемого газом. Как и во всех случаях использования физических моделей, применимость модели идеального газа к реальному газу зависит не только от свойств самого газа, но и от характера вопроса, на который требуется найти ответ. Такая модель не позволяет описать особенности поведения различных газов, но выделяет общие свойства для всех газов. Идеальный газ удовлетворяет условиям:
	\begin{itemize}
		\item объемом всех молекул газа можно пренебречь по сравнению с объемом сосуда, в котором находится газ
		\item время столкновения молекул друг с другом пренебрежимо мало по сравнению со временем между двумя столкновениями (временем свободного пробега)
		\item молекулы взаимодействуют между собой только при непосредственном соприкосновении, при этом они отталкиваются
		\item силы притяжения между молекулами идеального газа пренебрежимо малы
	\end{itemize}
	Используя модель идеального газа, выразим зависимость давления газа от средней кинетической энергии его молекул.\par
	Для начала определим среднее значение квадрата скорости молекул газа формулой
		$$\average{v^2}=\frac{\sum_{i=1}^N v_i}{N},$$ 
	где $N$ --- количество молекул газа в сосуде. Квадрат модуля вектора равен сумме квадратов его проекций на оси $Ox$, $Oy$ и $Oz$, поэтому
		$$\average{v^2}=\average{v_x^2}+\average{v_y^2}+\average{v_z^2}.$$
	Так как направления $Ox$, $Oy$ и $Oz$ вследствие беспорядочного движения молекул равноправны, средние значения квадратов проекций по осям равны:
		$$\average{v_x^2}=\average{v_y^2}=\average{v_z^2},$$
	откуда
		$$\average{v_x^2}=\frac{1}{3}\average{v^2}.$$
	%РИСУНОК!!!!!!!!!!!11111111111111111111
	Каждая молекула массой $m_1$, упруго ударяющаяся со скоростью $\vec{v}$ о стенку сосуда, площадь которой равна $S$, меняет свой импульс на $\Delta p = 2m_1v_x$. Пусть за время $dt$ о стенку ударяются молекулы из объема $dV=Sv_xdt$, тогда число молекул из этого объема равно $dN=ndV=nSv_xdt$, а молекул, подлетающих к стенке, --- 
		$$dN_+ \hm=\dfrac{1}{2}dN=\dfrac{1}{2}nSv_xdt.$$
	Полный импульс, переданный стенке за $dt$,  равен 
		$$dp_{\text{полн}}=\Delta p\,dN_+=\frac{1}{2}nSv_xdt \cdot 2m_1v_x=nm_1v_x^2Sdt,$$
	откуда, согласно второму закону Ньютона, сила, действующая на стену равна
		$$F=\frac{dp_{\text{полн}}}{dt}=nm_1v_x^2S.$$
	В действительности средняя за $dt$ сила, действующая на стенку, пропорциональна не $v_x^2$, а $\average{v_x^2}$:
		$$\average{F}=nm\average{v_x^2}S=\frac{1}{3}nm\average{v^2}.$$
	Таким образом, давление газа на стенку сосуда равно
	\begin{equation}\label{1}
		\boxed{P=\frac{\average{F}}{S}=\frac{1}{3}nm_1\average{v^2}.}
	\end{equation}
	Это и есть основное уравнение молекулярно-кинетической теории; оно связывает макроскопическую величину --- давление --- с микроскопическими величинами, характеризующими молекулы. \par
	Если теперь через $\average{E}_\text{кин}=\dfrac{m_0\average{v^2}}{2}$ обозначить среднюю кинетическую энергию поступательного движения молекулы, то уравнение ~(\ref{1}) можно переписать в виде
	\begin{equation}\label{2}
		\boxed{P=\frac{2}{3}n\average{E}.}
	\end{equation}
	Итак, \textit{\textbf{давление идеального газа пропорционально произведению концентрации молекул на среднюю кинетическую энергию поступательного движения молекулы}}.