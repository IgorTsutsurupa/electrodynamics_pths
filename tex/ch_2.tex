\section{Термодинамическое равновесие}

	В механике состояние системы в момент времени определяется заданием координат и импульсов всех входящих в нее частиц. Понимаемое в таком смысле состояние мы будем называть \textit{микроскопическим состоянием} системы или ее \textit{микросостоянием}. Наряду с ним можно рассматривать \textit{макроскопическое состояние} или \textit{макросостояние}, характеризуемое заданием только макроскопических параметров. Одному и тому же макросостоянию системы может соответствовать множество ее различных микросостояний. В термодинамике рассматриваются только макроскопические состояния, которые в дальнейшем будем называть просто состояниями. \par
	Состояние называется стационарным, если все макроскопические параметры системы не меняются со временем. Стационарное состояние может поддерживаться внешними по отношению к системе процессами. Например, неизменный во времени перепад температур между концами стержня можно создать, нагревая все время один его конец и охлаждая другой. \par
	Если стационарность состояния не обусловлена внешними процессами, то состояние называется равновесным или состоянием термодинамического равновесия. Когда макроскопическая система находится в состоянии термодинамического равновесия, то все макроскопические части, на которые можно мысленно или реально разбить эту систему, также находятся в равновесии как сами по себе, так и друг с другом.\par
	Изолированная система --- система, не обменивающаяся энергией и веществом с внешними телами.
	\begin{post}[на основе опытных фактов]
		Любая изолированная система рано или поздно приходит в состояние термодинамического равновесия и самопроизвольно из него выйти не может.
	\end{post}
	Опыт показывает, что в состоянии равновесия не все макроскопические параметры, которые можно использовать для описания системы, являются независимыми. Независимы только внешние параметры. Например, если некоторое количество газа находится в сосуде определенного объема при некоторой температуре, то его давление имеет вполне определенное значение, которое однозначно выражается через объем и температуру газа. Другими словами, для определенного количества газа давление является функцией объема и температуры.