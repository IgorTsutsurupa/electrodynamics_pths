\section{Адиабатический процесс}	

	\begin{defn}
		Адиабатический (адиабатный) процесс --- процесс изменения объема и давления газа, при отсутствии теплообмена с окружающим пространством, т.е. при условии $\delta Q=0$.
	\end{defn}
	Примерами адиабатных процессов могут служить процессы сжатия воздуха в цилиндре двигателя внутреннего сгорания.\par
	В соответствие с первым началом термодинамики при адиабатическом сжатии изменение внутренней энергии газа равно работе внешних сил
		$$dU = -\delta A,$$
	поэтому при адиабатическом расширении температура газа понижается. 	Поскольку при адиабатическом сжатии температура газа повышается, давление газа с уменьшением объема растет быстрее, чем при изотермическом процессе. Понижение температуры газа при адиабатическом расширении приводит к тому, что давление газа убывает быстрее, чем при изотермическом процессе.\par
	Только при адиабатическом и изотермическом процессах исключен контакт газа с телами иной температуры, при адиабатическомом он теплоизолирован, а при изотермическом он соприкасается только с термостатом, телом той же температуры. Значит, именно при этих процессах исключен переход внутренней энергии газа во внутреннюю энергию других тел, при котором работа не совершается, поэтому в этих двух процессах изменение внутренней энергии газа или термостата может быть полностью использовано для совершения работы.\par
	Работа при адиабатном процессе:
		$$A = -\Delta U = cm(T_1-T_2)$$
   	Для одного моля идеального газа работа равна 									            
    	$$A = C_v(T_1-T_2)$$
	Из уравнения состояния идеального газа для одного моля выразим температуру: 
		$$T = \frac{PV}{R}, \qquad \text{или} \qquad T = \frac{PV}{C_p-C_v}$$
	Имеем уравнение для работы газа при адиабатическом расширении: 
		$$A = Сv \frac{P_1V_1-P_2V_2}{C_p-C_v}$$
	Отношение теплоемкостей $\gamma = \dfrac{C_p}{C_v}$ нызвают коэффициентом Пуассона. Его можно выразить через число степеней свободы. Удельная теплоемкость при постоянном объеме: $C_v = \dfrac{iR}{2\mu}$. А удельная теплоемкость при постоянном давлении
		$$C_p = C_v + \frac{R}{\mu} = \frac{iR}{2\mu}+\frac{R}{\mu}  = \frac{R(\dfrac{i}{2}+1)}{\mu} ,$$
		$$\gamma = \frac{C_p}{C_v} = \frac{i+2}{i}$$
	Для одноатомного и двухатомного газов $i$ равно 3 и 5 соответственно, тогда для них $\gamma$ равна соответственно $\dfrac{3+2}{3} = \dfrac{5}{3}$ и $\dfrac{5+2}{5} = \frac{7}{5}$. Мы вывели 
		$$A = Cv \frac{P_1V_1-P_2V_2}{C_p-C_v}.$$
	Разделим числитель и знаменатель на $С_v$ и получим
		$$A = \frac{P_1V_1-P_2V_2}{\gamma-1}.$$
	Имеем окончательную формулу для работы при адиабатическом расширении. Теперь выведем уравнение адиабаты.
		$$ dU=-PdV$$ %(1)
	Приращение внутренней энергии $dU = Cm\cdot dT$, поскольку внутренняя энергия идеального газа не зависит от объема, выражения для $dT$ можно получить из уравнения состояния:
		$$PdV +VdP = RdT.$$
	Для $dT$ и $dU$ уравнение (1) переписывается в виде $C_v VdP=-C_p PdV$. 
	Мы воспользовались тем, что 
		$$Сv + R = Cp, \qquad \gamma  = \frac{C_p}{C_v}$$
	Разделив обе части на произведения $PV$ получаем дифференциальное уравнение выражающее зависимость давления идеального газа от объема в адиабатическом процессе:
		$$\frac{dP}{P} = -\gamma \frac{dV}{V}.$$
	Интегрируя, получаем
		$$PV^\gamma  = C.$$
	%Используя уравнение состояния можно записать иначе:
	%TV ^ γ -1 = C1
