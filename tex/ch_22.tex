\section{Распределение Больцмана}

	Выделим в изотермической атмосфере ($T=\c$) цилиндрический столбик, в котором находится $\nu$ моль молекул идеального газа, и отложим ось $Oz$ перпендикулярно поверхности земли. Как в этом случае изменяется давление с изменением высоты? Ясно, что у поверхности земли $P_0=P(0)$. Вспомним, что высота линейно зависит от концентрации и плотности:
		$$P=nkT=\dfrac{\rho}{\mu}RT,$$ 
	тогда зависимость $n=n(h)$ и $\rho=\rho(h)$ будет такой же, как и $P=P(h)$.\par
	Рассмотрим на высоте $h$ слой газа толщины $dh$. Тогда
		$$dP=P(h+dh)-P(h)=-\frac{d(mg)}{S}=\rho gSdh.$$
	С другой стороны
		$$dP=\frac{d\rho}{\mu}RT,$$
	откуда
		$$\frac{RT}{\mu}d\rho=-\rho gdh,$$
	или
	\begin{equation}\label{foo}
		\frac{d\rho}{\rho}=-\frac{\mu g}{RT}dh.
	\end{equation}
	Интегрируя обе части равенства и избавляясь от логарифма, получим
		$$\rho=\rho_0\exp{\left(-\dfrac{\mu gh}{RT}\right)}.$$
	Аналогично выглядит формула для концентрации и давления. Можно обобщить эту формулу на случай иной зависимости потенциальной энергии $U$ от высоты:
	\begin{equation}\label{boltzmann}
		\quad P(h)=P_0\exp{\left(-\dfrac{U(h)}{RT}\right)}.
	\end{equation}
	Эту функцию называют распределением Больцмана.\par
	Если $T=T(h)$, то, в соответствии с ~(\ref{foo}),
	\begin{equation}
		P(h)=P_0\exp{\left(-\frac{U(h)}{k}\int \frac{dh}{T(h)}\right)}
	\end{equation}
