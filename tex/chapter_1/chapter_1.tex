\section{Микро- и макрохарактеристики. Соотношения между ними}

Молекулярная физика --- раздел физики, изучающий молекулярное строение вещества. \par
Макрохарактеристики --- параметры тела, не требующие знаний о молекулярном строении вещества для своего описания. \par
Микрохарактеристики --- параметры тела, существенно использующие молекулярную структуру для своего описания. \par
Основные постулаты молекулярно-кинетической теории (МКТ):
\begin{enumerate}
	\item В большинстве случаев вещество состоит из огромного числа микроскопических структурных частиц --- молекул или атомов (в условиях, не сильно отличающихся от нормальных).
	\item Все частицы находятся в непрерывном хаотическом движении.
\end{enumerate}

\begin{table}[htp]
	\centering
	\begin{tabular}{*{6}{c}}
		\toprule
			Макро & Микро & Соотношения \\
		\midrule
			масса тела $M$ 		& масса частицы $m_0$ 	& $M=m_0N$ 									\\
			объем $V$ 			& число частиц $N$ 		& $n=\dfrac NV$ 							\\
			плотность $\rho$	& концентрация $n$		& $\rho=\dfrac MV=\dfrac{m_0 N}V = m_0n$ 	\\
								& молярная масса $\mu$	& $\mu=m_0N_A$ 								\\
			
		\toprule
	\end{tabular}
\end{table}

Нагретость --- свойство тела создавать тепло. \par
Макроскопическая температура $T_{\text{макро}}$ --- мера нагретости тела. \par
Микроскопическая температура $T_{\text{микро}}$ --- мера средней кинетической энергии поступательного движения молекул.