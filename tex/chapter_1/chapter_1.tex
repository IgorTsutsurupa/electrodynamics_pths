\section{Микро- и макрохарактеристики. Соотношения между ними}

Молекулярная физика --- раздел физики, изучающий микроскопическое движение молекул и эффекты, с этим связанные. \par
Макрохарактеристики --- параметры тела, не требующие знаний о молекулярном строении вещества для своего описания. Например, $M$ -- масса тела, $V$ -- его объем. \par
Микрохарактеристики --- параметры тела, существенно использующие молекулярную структуру для своего описания. \par
Основные постулаты молекулярно-кинетической теории (МКТ):
\begin{enumerate}
	\item В большинстве случаев вещество состоит из огромного числа микроскопических структурных частиц --- молекул или атомов (в условиях, не сильно отличающихся от нормальных).
	\item Все частицы находятся в непрерывном хаотическом движении.
\end{enumerate}

\begin{table}[htp]
	\caption{Соотношения между макро- и микрохарактеристиками}
	\centering
	\begin{tabular}{c|c|c}
		\toprule
			Макро- & Микро- & Соотношения \\
		\midrule
			масса тела $M$					& масса частицы $m_1$			 & $M=m_1 N$											\\
			объем $V$						& число частиц $N$				 & $n=\dfrac {N}{V}$									\\
			плотность $\rho$				& концентрация $n$				 & $\rho=\dfrac {M}{V}=\dfrac{m_1 N}{V} = m_1n$			\\
			кол-во вещества $\nu$			&								 & $\nu=\dfrac {N}{N_A}$								\\
			молярная масса $\mu$			& 								 & $\mu=m_1 N_A$										\\
			температура $T_{\text{макро}}$	& температура $T_{\text{микро}}$ & $\dfrac{3}{2}kT_{\text{микро}}=\langle E_1 \rangle$	\\
											& ср. кин. энергия частицы $\langle E_1 \rangle$ & $\langle E_1 \rangle= \langle \dfrac{m_1 v^2}{2} \rangle$ \\
%											& среднее расстояние между молекулами $a$ & \\
											& длина св. пробега $\lambda$	 & \\
											& время св. пробега $\tau$		 & \\
											& радиус частицы $r_0$			 & \\

		\toprule
	\end{tabular}
\end{table}
Величина $k=1,38 \times 10^{-23} \,\text{Дж}/\kelv$ называется постоянной Больцмана, $N_A = 6,02 \times 10^{-23} \,\text{моль}^{-1}$ --- постоянной Авогадро.\par
Нагретость --- свойство тела создавать тепло. \par
Макроскопическая температура $T_{\text{макро}}$ --- мера нагретости тела. \par
Микроскопическая температура $T_{\text{микро}}$ --- мера средней кинетической энергии поступательного движения молекул. Можно считать, что макроскопическая температура --- однозначная функция микроскопической.